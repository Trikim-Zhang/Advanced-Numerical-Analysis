\section{2023春考试题目}
% 第一题
\begin{example}
    已知$f(1)=5,f^{\prime}(1)=7,f^{\prime}(0)=1$ ,试问满足上插值述条件的多项式是否适定?若不适定请说明理由;若适定,则求出该多项式并给出误差估计.\Stars{5}
\end{example}
\begin{solution}
    设插值多项式$p(x) = a_0+a_1x+a_2x^2$满足题述条件,那么有
    \[
        \left\{
            \begin{array}{l}
                a_0+a_1+a_2 = 5\\
                a_1+2a_2 = 7\\
                a_1 = 1
            \end{array}
        \right.
    \]
    解得$a_0 = 1,\,a_1 = 1,\,a_2 = 3$,方程组有唯一解,插值多项式适定。
    
    设插值误差为$R(x)$,并做$\varphi(t) = f(t)-p(t)-W(t)$满足$\varphi(x) = 0$,即
    \[
        W(x) =  R(x) 
    \]
    
    因为$R(x)$满足$R(1) = R'(1) = 0$,故而设$W(t) = (mt+n)(t-1)^2$,由$W'(0) = 0$可得
    \[
        m(0-1)^2+(m\cdot 0+n)\cdot 2\cdot (0-1) = 0
    \]
    解得$m = 2n$

    两次利用罗尔定理可得
    \[
        \varphi'''(\xi) = f'''(\xi)- 3!m = 0
    \]
    解得$m = \dfrac{f'''(\xi)}{6}$。故而
    \[
        R(x) = \frac{f'''(\xi)}{12}(x-1)^2(2x+1)
    \]
\end{solution}
% 第二题
\begin{example}
    设$f(x)=0$有单根$x^*$,$x=\varphi(x)$是$f(x)=0$的等价方程,若$\varphi(x)=x-m(x)f(x)$,且所有函数都充分光滑。证明:当$m(x^*)\neq\frac1{f^{\prime}(x^*)}$时, $x_{k+1}=\varphi(x_k)$至多是一阶收敛的;当$m(x^*)=\frac1{f^{\prime}(x^*)}$时,$x_{k+1}=\varphi(x_k)$至少是二阶收敛的。\Stars{5}
\end{example}
\begin{proof}
    % 容易知道
    % \[
    %     \begin{array}{l}
    %         x^{*} = \varphi(x^{*}) = x^*-m(x^*)f(x^*) \\
    %         x_{k+1} = \varphi(x_{k}) = x_{k}-m(x)f(x)
    %     \end{array}
    % \]
    % $\varphi(x_{k}$)在$x^*$处的泰勒展开为
    % \[
    %     x_{k+1} = \varphi(x^*) + \varphi'(x^*)(x_{k}-x^*)+\frac{\varphi''(x^*)}{2}(x_{k}-x^*)^2+\cdots
    % \]
    % 有
    % \[
    %     \begin{aligned}
    %         \frac{|x_{k+1}-x^{*}|}{|x_{k}-x^{*}|} &= \frac{|\varphi(x^*)-\varphi(x^*) - \varphi'(x^*)(x_{k}-x^*)-\frac{\varphi''(x^*)}{2}(x_{k}-x^*)^2+\cdots|}{|x_{k}-x^{*}|}\\
    %         &= \left|\varphi'(x^*) + \frac{\varphi''(x^*)}{2}(x_{k}-x^*)+\cdots\right|
    %     \end{aligned}
    % \]
    % $\varphi'(x^*)$为
    % \[
    %     \varphi'(x^*) = 1-m'(x^*)f(x^*)-m(x^*)f'(x^*)
    % \]
    % 当$m(x^*)\neq\frac1{f^{\prime}(x^*)}$时,
    % \[
    %     \varphi'(x^*) = 1-m'(x^*)f(x^*)-m(x^*)f'(x^*) =1-m(x^*)f'(x^*) \neq 0
    % \]
    % 则
    % \[
    %     \begin{aligned}
    %         \frac{|x_{k+1}-x^{*}|}{|x_{k}-x^{*}|} = \left|\varphi'(x^*) + \frac{\varphi''(x^*)}{2}(x_{k}-x^*)+\cdots\right| \to \left| \varphi'(x^*)  \right|,(k\to \infty)
    %     \end{aligned}
    % \]
    % 当$m(x^*)\neq\frac1{f^{\prime}(x^*)}$时, $x_{k+1}=\varphi(x_k)$至多是一阶收敛的;
    % \newline
    % 当$m(x^*)=\frac1{f^{\prime}(x^*)}$时,
    % \[
    %     \varphi'(x^*) = 1-m'(x^*)f(x^*)-m(x^*)f'(x^*) = 0
    % \]
    % 则
    % \[
    %     \begin{aligned}
    %         \frac{|x_{k+1}-x^{*}|}{|x_{k}-x^{*}|^2} = \left|\frac{\varphi''(x^*)}{2}+\frac{\varphi'''(x^*)}{3!}(x_{k}-x^*)+\cdots\right| \to \left| \frac{\varphi''(x^*)}{2} \right|,(k\to \infty)
    %     \end{aligned}
    % \]
    % 当$m(x^*)=\frac1{f^{\prime}(x^*)}$时,$x_k+1=\varphi(x_k)$至少是二阶收敛的
    由$x^*$是$f(x) = 0$的单根,有
    \[
        f(x^*) = 0,\,f'(x^*)\neq 0
    \]
    由$\varphi(x) = x-m(x)f(x)$,有
    \[
        \begin{array}{c}
            \varphi'(x) = 1-m'(x)f(x)-m(x)f'(x)\\
            \varphi'(x^*) = 1-m'(x)^*f(x^*)-m(x^*)f'(x^*) = 1-m(x^*)f'(x^*)           
        \end{array}
    \]
    由迭代阶的定理,当$m(x^*)\neq 1/f'(x^*)$时,$\varphi'(x^*) = 1-m(x^*)f'(x^*) \neq 0$.此时若$|\varphi'(x^*)|<1$,则迭代法$x_{k+1}=\varphi(x_k)$一阶收敛;若$|\varphi'(x^*)|\geq 1$,则迭代法$x_{k+1}=\varphi(x_k)$不收敛;故迭代法是最多是一阶收敛的。

    $m(x^*)= 1/f'(x^*)$时,$\varphi'(x^*) = 1-m(x^*)f'(x^*) = 0$.此时则迭代法$x_{k+1}=\varphi(x_k)$至少是二阶收敛的.
\end{proof}
% 第三题
\begin{example}
    对下列线性代数方程组给出使 Jacobi 迭代法和 Gauss-Seidel 迭代法均收敛的迭代格式,要求分别写出这两个迭代格式,并说明迭代法收敛的理由。\Stars{5}
    \[
        \left\{
            \begin{array}{l}
                2x_1-x_2+x_4=1\\
                x_1-x_3+5x_4=6\\
                x_2+4x_3-x_4=8\\
                -x_1+3x_2-x_3=3
            \end{array}
        \right.
    \]
\end{example}

\begin{solution}
    \[
        \begin{bmatrix}
            \boldsymbol{A} \mid \boldsymbol{b}
        \end{bmatrix}=
        \begin{bmatrix}
            2&-1&0&1&1\\1&0&-1&5&6\\0&1&4&-1&8\\-1&3&-1&0&3
        \end{bmatrix}
        \xrightarrow{\stackrel{r_2\leftrightarrow r_4}{r_1\times 10+r_2}}
        \begin{bmatrix}
            19&-7&-1&10&10\\-1&3&-1&0&3\\0&1&4&-1&8\\1&0&-1&5&6
        \end{bmatrix}
    \]
    因其变化后为等价方程组,且严格对角占优,故而Jacobi 迭代法和 Gauss-Seidel 迭代法均收敛。

    Jacobi 迭代格式为:
    \[
        \begin{aligned}
            \begin{cases}
                x_1^{(m+1)}=\frac{1}{19}(7x_2^{(m)}+x_3^{(m)}-10x_4^{(m)}+10)\\
                x_2^{(m+1)}=\frac{1}{3}(x_1^{(m)}+x_3^{(m)}+3)\\
                x_3^{(m+1)}=\frac{1}{4}(-x_2^{(m)}+x_4^{(m)}+8)\\
                x_4^{(m+1)}=\frac{1}{5}(-x_1^{(m)}+x_3^{(m)}+6)
            \end{cases}
            & (m=0,1,2,\cdots)
        \end{aligned}  
    \]
    Gauss-Seidel迭代格式:
    \[
        \begin{aligned}
            \begin{cases}
                x_1^{(m+1)}=\frac{1}{19}(7x_2^{(m)}+x_3^{(m)}-10x_4^{(m)}+10)\\
                x_2^{(m+1)}=\frac{1}{3}(x_1^{(m+1)}+x_3^{(m)}+3)\\
                x_3^{(m+1)}=\frac{1}{4}(-x_2^{(m+1)}+x_4^{(m)}+8)\\
                x_4^{(m+1)}=\frac{1}{5}(-x_1^{(m+1)}+x_3^{(m+1)}+6)
            \end{cases}
            & (m=0,1,2,\cdots)
        \end{aligned}  
    \]
\end{solution}
% 第四题
\begin{example}
    求$f\left(x\right)=3x^3+2x^2+x$在区间$\left[-1,1\right]$上的2次最佳一致逼近多项式。\Stars{5}
\end{example}
\begin{solution}
    切比雪夫多项式为
    \[
        \begin{array}{l}
            T_{0}(x) = 1\\
            T_{1}(x) = x\\
            T_{2}(x) = 2x^2-1\\
            T_{3}(x) = 2xT_{2}(x)-T_{1}(x) = 4x^3-3x
        \end{array}
    \]
    有
    \[
        \begin{array}{l}
            \dfrac{f(x)-p(x)}{3} = \dfrac{T_{3}(x)}{4}\\
            \Rightarrow p(x) = 2x^2+\dfrac{13}{4}x
        \end{array}
    \]
\end{solution}
% 第五题
\begin{example}
    已知$f(-1)=1,f(-0.5)=4,f(0)=6,f(0.5)=9,f(1)=2,$且$\mid f^{(4)}(x)\mid\leq M(\forall x\in[-1,1])$,$M$为常数,试估计用复合 Simpson 公式计算积分$\int_{-1}^{1}f(x)\diff x$的整体截断误差限.\Stars{5}
\end{example}
\begin{solution}
    因为
    \[
        \begin{aligned}
            \int_{-1}^{1}f(x)dx& =\int_{-1}^{0}f(x)\diff x+\int_{0}^{1}f(x)\diff x  \\
            &\approx[\frac{1}{6}f(-1)+\frac{4}{6}f(-0.5)+\frac{1}{6}f(0)]+[\frac{1}{6}f(0)+\frac{4}{6}f(0.5)+\frac{1}{6}f(1)]\\
            &\approx \frac{1}{6}[1+4\times4+6+6+4\times9+2]=\frac{67}{6}\approx 11.1667
        \end{aligned}
    \]
    误差
    \[
        \begin{aligned}
            \left|I-S_{2}\right| & \leq\left|\int_{-1}^{0}\frac{f^{(4)}(\xi_{1})}{4!}(x+1)(x+0.5)^{2}(x-0)\diff x\right|+\left|\int_{0}^{1}\frac{f^{(4)}(\xi_{2})}{4!}(x-0)(x-0.5)^{2}(x-1)\diff x\right|\\
            &\leq \frac{M}{24}\left[ \int_{-1}^{0}\Bigl|(x+1)(x+0.5)^{2}(x-0)\Bigr|\diff x+\int_{0}\Bigl|(x-0)(x-0.5)^{2}(x-1)\Bigr|\diff x \right]\\
            &\leq \frac{M}{12}\int_{0}^{1}\Bigl|(x-0)(x-0.5)^{2}(x-1)\Bigr|\diff x=\frac{M}{6}\int_{0}^{0.5}t^{2}(0.25-t^{2})\diff t=\frac{M}{6}\times0.0042\\
            & \leq 0.008M 
        \end{aligned}
    \]
\end{solution}
% 第六题
\begin{example}
    设$n$阶实矩阵$A$的特征值满足:$\lambda_1=\lambda_2,\lambda_1>|\lambda_3|\geq\cdots\geq|\lambda_n|$,试讨论用幂法如何求主特征值,并分析收敛性.
\end{example}
% 第八题
\begin{example}
    设多元函数$G:D\subset\mathbb{R}^n\to\mathbb{R}^n,\forall \boldsymbol{x},\boldsymbol{y}\in D_0\subset D$都有$\|G(\boldsymbol{x})-G(\boldsymbol{y})\|\leq\|\boldsymbol{x}-\boldsymbol{y}\|$,则称$G$在$D_0$为非膨胀映射,若当$\boldsymbol{x}\neq \boldsymbol{y}$时不等式严格成立,则称$G$在$D_0$为严格非膨胀映射.

    假定$G{:}D\subset\mathbb{R}^n\to\mathbb{R}^n$,在有界闭集$D_0\subset D$上是严格非膨胀映射,且$G(D_0)\subset D_0$,证明$G$在$D_0$中有唯一不动点.\Stars{5}
\end{example}
\begin{proof}
    设$\varphi(\boldsymbol{x}) = \|\boldsymbol{x}-\boldsymbol{G}(\boldsymbol{x})\|$
    
    显然,设$\varphi(\boldsymbol{x}) = \|\boldsymbol{x}-\boldsymbol{G}(\boldsymbol{x})\|$在$D_0$上连续,而$D_0$为闭集,故$\varphi(\boldsymbol{x})$在$D_0$上有最小值,记为$\varphi(\boldsymbol{x}^*) = \min\limits_{\boldsymbol{x}\in D_0}\|\boldsymbol{x}-\boldsymbol{G}(\boldsymbol{x})\|$

    若$\boldsymbol{G}(\boldsymbol{x}^*)\neq \boldsymbol{x}^*$,则有
    \[
        \varphi(\boldsymbol{G}(\boldsymbol{x}^*)) = \|\boldsymbol{G}(\boldsymbol{x}^*)-\boldsymbol{G}(\boldsymbol{G}(\boldsymbol{x}^*))\|<\|\boldsymbol{x}^*-\boldsymbol{G}(\boldsymbol{x}^*)\|
    \]矛盾!
    
    所以必有$\boldsymbol{G}(\boldsymbol{x}^*) = \boldsymbol{x}^*$。
\end{proof}