% !TeX root = ../elegantnote.tex
\section{微分方程数值解的基本概念(没有大题)}
\subsection{微分方程数值解的基本概念}
\begin{theorem}
    如果方程$\begin{cases}\frac{dy}{dx}=f(x,y),\\y(x_0)=y_0.\end{cases}$中右端函数$f(x,y)$满足
    \begin{enumerate}
        \item $f(x,y)$是实值函数
        \item $f(x,y)$在矩形区域$\Omega=\{(x,y)|x\in[x_{0},T],y\in(-\infty,+\infty)\}$内连续
        \item $f(x,y)$关于$y$满足Lipschitz条件:即存在正常数$L$,使得对任意$x\in\left[ x_0,T \right]$,均成立不等式
        \[
            |f(x,y_1)-f(x,y_2)|\leq L|y_1-y_2|
        \]
    \end{enumerate}
    则方程存在唯一解$y(x)\in C^{1}[x_0,T]$
\end{theorem}
\begin{definition}[扰动问题]
    考虑扰动问题$\begin{cases}\dfrac{\mathrm{d}y}{\mathrm{d}x}=f(x,y)+\delta(x),x_0\leq x\leq T,\\y(x_0)=y_0+\varepsilon.\end{cases}$
\end{definition}
\begin{example}
    以下初值问题是否关于数据的扰动稳定?
    \[
        \begin{cases}
            \dfrac{\mathrm{d}u}{\mathrm{d}t}=100u-101e^{-t}\\
            u(0)=1
        \end{cases}
    \]
    \begin{enumerate}
        \item [\choice{}{A}] 是
        \item [\choice{1}{B}] 否 
    \end{enumerate}
    \begin{solution}
        原问题的通解为
        \[
            \begin{aligned}
                u(t) &= e^{-\int-100\mathrm{d}t}\left( \int -101e^{-t}e^{\int-100\mathrm{d}t}\mathrm{d}t + C \right)\\
                &=e^{100t}\left( e^{-101t} + C \right)
            \end{aligned}
        \]
        代入初始条件$u(0) = 1$得到
        \[
            u(t) = e^{-t}
        \]
        现在有初值的扰动$u(0) = 1+\varepsilon$得到
        \[
            u(t) = e^{-t} + \varepsilon e^{100t}
        \]
        显然对$\forall \varepsilon>0,u(t;\varepsilon)$偏离真解很大,故问题不稳定。
    \end{solution}
\end{example}
\subsection{Euler方法}
\begin{definition}[Euler方法]
    考虑问题
    \[
        \begin{cases}
            \dfrac{\mathrm{d}y(x)}{\mathrm{d}x}=f(x,y(x)),x_0\leq x\leq T,\\
            y(x_0)=y_0.
        \end{cases}
    \]
    将区域$\left[ x_0,T \right]$剖分成$m$等份,步长$h = \dfrac{T-x_0}{m},$网点$x_{n}=x_{n-1}+h=x_{0}+(n-1)h,n=1,2,\cdots,m.$
\end{definition}

\begin{note}
    梯形法
    \[
        y(x_{n+1})=y(x_{n})+\frac{h}{2}[f(x_{n},y(x_{n}))+f(x_{n+1},y(x_{n+1}))]+R_{n}^{(1)}
    \]
    \[
        R_{n}^{(1)}=-\frac{h^{3}}{12}y^{\prime\prime\prime}(\xi),x_{n}<\xi<x_{n+1}
    \]
    \[
        y_{n+1}=y_{n}+\frac{h}{2}[f(x_{n},y_{n})+f(x_{n+1},y_{n+1})]
    \]
    \begin{itemize}
        \item 梯形法为二阶方法
        \[
            \mid y(x_n)-y_n\mid=O(h^2),\text{当}h\to 0\text{时}  
        \]
        \item 梯形法为单步隐式方法
        \item 梯形法计算格式
        \[
            \begin{cases}
                y_{n+1}^{(0)}=y_{n}+hf(x_{n},y_{n}),\\
                y_{n+1}^{(m+1)}=y_{n}+\dfrac{h}{2}[f(x_{n},y_{n})+f(x_{n+1},y_{n+1}^{(m)})],\\
                m=0,1,2,\cdots
            \end{cases}
        \]
        \item \colorbox{cyan!50}{梯形格式收敛的条件}是什么?
        \[    
            y_{n+1}^{*} = y_n+\dfrac{h}{2}\left[ f(x_n,y_n)+f(x_{n+1},y_{n+1}^{*}) \right]
        \]
        记$y_{n+1}^{(m)}-y_{n+1}^{*} = \varepsilon^{m}$,则
        \[
            \begin{aligned}
                |\varepsilon^{(m+1)}| &= \dfrac{h}{2}\left[ f(x_{n+1},y_{n+1}^{(m)})+f(x_{n+1},y_{n+1}^{*}) \right]\\
                &\leqslant \dfrac{hL}{2}|y_{n+1}^{(m)}-y_{n+1}^{*}| = \dfrac{hL}{2}\varepsilon^{(m)}\\
                &=\left( \dfrac{hL}{2} \right)^{m}\varepsilon^{(0)}
            \end{aligned}
        \]
        \[
            \lim_{m\to \infty}|\varepsilon^{(m+1)}|\Longrightarrow\colorbox{cyan!50}{$\dfrac{hL}{2}<1$}
        \]
    \end{itemize}
\end{note}
\begin{example}
    下列关于梯形法的叙述错误的是\Stars{3}{}
    \begin{enumerate}
        \item [\choice{1}{A}]梯形法是一个显式方法
        \item [\choice{1}{B}]梯形法的整体截断误差是$O(h^3)$
        \item [\choice{}{C}]梯形法是一个稳定的方法
        \item [\choice{1}{D}]梯形法的迭代步长可以任取
    \end{enumerate}
\end{example}
\begin{note}
    改进Euler法
    \[
        \begin{cases}
            \overline{y}_{n+1}=y_n+hf(x_n,y_n)&\text{\textcolor{red}{预测}}\\
            y_{n+1}=y_n+\frac{h}{2}[f(x_n,y_n)+f(x_{n+1},\overline{y}_{n+1})]&\text{\textcolor{red}{校正}}
        \end{cases}
    \]
    预测校正格式还可写为
    \[
        y_{n+1}=y_n+\dfrac{h}{2}\Big[f(x_n,y_n)+f(x_n+h,y_n+hf(x_n,y_n))\Big]
    \]
    \[
        \begin{cases}
            y_p=y_n+hf(x_n,y_n)\\
            y_c=y_n+hf(x_{n+1},y_p)\\
            y_{n+1}=\dfrac{1}{2}[y_n+y_p]
        \end{cases}
    \]
\end{note}
\begin{example}
    改进Euler法是\sol{单} 步 (填``单''或``多'')\sol{显} 格式 (填``显''或``隐'') 格式\Stars{3.5}{}
\end{example}
\subsection{Runge-Kutta方法}
\begin{definition}[Taylor级数法]
    \[
        y(x_{0}+h) =y(x_0)+hy'(x_0)+\frac{h^2}2y''(x_0)+\cdots+\frac{h^{q}}{q!}y^{(q)}(x_{0})+O(h^{q+1}) 
    \]
    \[
        \begin{aligned}
            y'(x_0) &= \dfrac{\mathrm{d}}{\mathrm{d}x}f(x,y)\big|_{x = x_0}=(f'_x+f'_y\dfrac{\mathrm{d}y}{\mathrm{d}x})\big|_{x = x_0}\\
            & = f'_x(x_0,y_0) + f'_y(x_0,y_0)f(x_0,y_0)
        \end{aligned}
    \]
    \[
        y_{n+1}=y_n+hy_n^{\prime}+\frac{h^2}2y_n^{\prime\prime}+\cdots+\frac{h^q}{q!}y_n^{(q)}
    \]
\end{definition}
\begin{note}
    Taylor级数法优缺点:
    \begin{itemize}
        \item 优点:可达任意阶精度
        \item 缺点:求导复杂,不实用
    \end{itemize}
\end{note}
\begin{note}
    Runge-Kutta方法的基本思想
    \[
        \frac{y(x_{n+1})-y(x_n)}h=y'(x_n+\theta h)
    \]
    \[
        \begin{aligned}
             y(x_{n+1})-y(x_n)&=\int_{x_{n}}^{x_{(n+1)}}f(x,y)\mathrm{d}x\\
             &=f(x_{n}+\theta h,y(x_n+\theta h))h
        \end{aligned}
     \]
\end{note}
\begin{definition}[Runge-Kutta法]
    设$s$是一个正整数代表使用函数值$f$的个数,$\alpha_i,\beta_{ij}$和$W_{i}$是一些待定的权因子 (为实数) ,方法
    \[
        y_{n+1}=y_n+h\sum_{i=1}^sW_iK_i
    \]
    其中$K_i$满足下列方程:
    \[
        \begin{aligned}
            &K_{1}=f(x_{n},y(x_{n})),K_{2}=f(x_{n}+\alpha_{2}h,y(x_{n})+h\beta_{21}K_{1})\\
            &K_{3}=f(x_{n}+\alpha_{3}h,y(x_{n})+h\beta_{31}K_{1}+h\beta_{32}K_{2})\\
            &\cdots\cdots\\
            &K_{s}=f(x_{n}+\alpha_{s}h,y(x_{n})+h\sum_{j=1}^{s-1}\beta_{sj}K_{j})
        \end{aligned}
    \]
    被称为一阶常微分方程的$s$级\textcolor{red!50}{显式}Runge-Kutta方法。
\end{definition}
\begin{note}
    Runge-Kutta方法可采用Butcher表表示
    \[
        \begin{aligned}
            &y_{n+1}=y_{n}+h\sum_{i=1}^{s}W_{i}K_{i} \\
            &K_{1}=f(x_{n},y(x_{n})) \\
            &K_{i}=f(x_{n}+\alpha_{i}h,y(x_{n})+h\beta_{i1}K_{1}+\cdots+h\beta_{i,i-1}K_{i-1})
        \end{aligned}
    \]
    % Table generated by Excel2LaTeX from sheet 'Butcher'
    \begin{table}[htbp]
        \centering
        \begin{tabular}{c|ccccc}
        $\alpha_{1}$ &     &     &     &     &  \\
        $\alpha_{2}$ & $\beta_{21}$ &     &     &     &  \\
        $\alpha_{3}$ & $\beta_{31}$ & $\beta_{32}$ &     &     &  \\
        $\cdots$ & $\cdots$ & $\cdots$ & $\cdots$ &     &  \\
        $\alpha_{s}$ & $\beta_{s1}$ & $\beta_{s2}$ & $\cdots$ & $\beta_{s,s-1}$ &  \bigstrut[b]\\
        \hline
            & $W_1$ & $W_2$ & $\cdots$ & $W_{s-1}$ & $W_{s}$ \bigstrut[t]\\
        \end{tabular}%
    \end{table}%
\end{note}
\begin{example}
    4级Runge-Kutta方法的局部截断误差是 \sol{5} 阶.
\end{example}
\begin{note}
    s级Runge-Kutta法的精度
    \begin{itemize}
        \item 当$\leqslant 4$时,$s$级RK方法的阶$q(s)=s$;
        \item 当$s=5,6,7$时,$s$级RK方法的阶$q(s)=s-1$;
        \item 当$s=8,9$时,$s$级RK方法的阶$q(s)=s-2$;
        \item $q(10)<8$.
    \end{itemize}
\end{note}
\textcolor{red}{以上都是单步法的代表。}
\begin{note}
    一般单步法:显式单步法的一般形式
    \[
        y_{n+1}=y_n+h\varphi(x_n,y_n,h)
    \]
    \begin{itemize}
        \item Euler法
        \[
            \varphi(x,y,h)=f(x,y)
        \]
        \item Taylor级数法
        \[
            \varphi(x,y,h)=\sum_{j=1}^k\frac{h^{j-1}}{j!}y^{(j)}(x)
        \]
        \item s级四阶RK方法
        \[
            \varphi(x,y,h)=\frac{1}{6}(K_1+2K_2+2K_3+K_4)
        \]
    \end{itemize}
\end{note}
\begin{definition}[收敛]
    若对于任意的初值$y_0$及任意$x\in[x_0,T]$
    \[
        \lim_{h\to0}y_n=y(x_n),
    \]
    称$\varepsilon(x,y,h)$确定的单步方法是收敛的
\end{definition}
\begin{definition}[单步法的阶数]
    单步法
    \[
        y_{n+1}=y_n+h\varphi(x_n,y_n,h)
    \]
    称为\colorbox{cyan!50}{$p$阶}的,是指:对于真解$y(x),p$是使关系式
    \[
        y(x+h)-y(x)=h\varphi(x,y(x),h)+O(h^{p+1})
    \]
    成立的最大整数.
\end{definition}
\begin{theorem}
    假设单步法具有$p$阶精度,且增量函数$\varepsilon(x,y,h)$ 关于满足Lipschitz条件:
    \[
        \mid\varphi(x,y,h)-\varphi(x,\overline{y},h)\mid\leq L_\varphi(y-\overline{y})
    \]
    又设初值$y_0= y(x)$是准确的,则整体截断误差为
    \[
        y(x_n)-y_n=O(h^p)
    \]
\end{theorem}
\begin{proof}
    \[
        \begin{cases}
            y_{n+1} = y_n + \varphi(x_n,y_n,h)\\
            y(x_{n+1}) = y(x_{n})+h\varphi(x_{n},y(x_{n}),h) + Ch^{p+1}
        \end{cases}
    \]
    令$\varepsilon_{n} = y(x_{n})-y_{n}$,那么有
    \[
        \begin{aligned}
            |\varepsilon_{n+1}| & \leqslant |\varepsilon_{n}| + h|\left[\varphi(x_{n},y(x_{n}),h) -\varphi(x_n,y_n,h) \right]| + Ch^{p+1}\\
            &\leqslant |\varepsilon_{n}| + hL|y(x_{n})-y_n| + Ch^{p+1}\\
        \end{aligned}
    \]
    所以
    \[
        \begin{aligned}
            |\varepsilon_{n+1}| & \leqslant (1+hL)|\varepsilon_{n}| + Ch^{p+1}\\
            &\leqslant(1+hL)|^2\varepsilon_{n-1}| + \left[ (1+hL)+1 \right]Ch^{p+1}\\
            &\leqslant (1+hL)^{n}|\varepsilon_0| + \left[(1+hL)^n+(1+hL)^{n-1} + \cdots + 1  \right]Ch^{p+1}\\
            &\leqslant (1+hL)^{n}|\varepsilon_0| + \dfrac{1}{\cancel{h}L}Ch^{p\cancel{+1}}
        \end{aligned}
    \]
    证毕!
\end{proof}
\begin{example}
    分析改进Euler法的收敛性\Stars{3}{}
    \[
        \begin{cases}
            \overline{y}_{n+1}=y_n+hf(x_n,y_n)\\
            y_{n+1}=y_n+\frac{h}{2}[f(x_n,y_n)+f(x_{n+1},\overline{y}_{n+1})]
        \end{cases}
    \]
    \begin{solution}
        \[
            \varphi(x,y,h) = \dfrac{1}{2}\left[ f(x,y)+f(x+h,y+hf(x,y)) \right]
        \]
        那么有
        \[
            \begin{aligned}
                & |\varphi(x,y,h) -\varphi(x,\bar{y},h) |\\
                & \leqslant \dfrac{1}{2}|f(x,y)-f(x,\bar{y})| + \dfrac{1}{2}\left|f(x+h,y+hf(x,y))-f(x+h,\bar{y}+hf(x,\bar{y}))\right|\\
                &\leqslant \dfrac{L}{2}\left|y-\bar{y}\right|+\dfrac{L}{2}|(y-\bar{y})+h\left[ f(x,y)-f(x,\bar{y}) \right]|\\
                &\leqslant (L+\dfrac{L^2h}{2})|y-\bar{y}|
            \end{aligned}
        \]
    \end{solution}
\end{example}
\subsection{单步法的收敛性与稳定性}
对于模型问题$y' =\lambda y$,考察格式的稳定性,有$y = Ce^{\lambda x}$
\begin{itemize}
    \item Euler法$y_{n+1}=y_n+hf(x_n,y_n)$(\colorbox{red!50}{条件稳定})
    
    根据$\varepsilon_n = y_n-z_n$有
    \[
        \begin{array}{l}
            y_{n+1} = y_n + h\lambda y_n\\
            z_{n+1} = z_n + h\lambda z_n
        \end{array}
    \]
    \[
        \begin{aligned}
            \varepsilon_{n+1}=(1+h\lambda)\varepsilon_n
        \end{aligned}
    \]
    \item 后退Euler法$y_{n+1}=y_n+hf(x_{n+1},y_{n+1})$(\colorbox{cyan!50}{无条件稳定})
    \[
        \begin{aligned}
            y_{n+1} = y_n+h\lambda y_{n+1}\Rightarrow  y_{n+1} =\dfrac{1}{1-\lambda h}y_n
        \end{aligned}
    \]
    那么有
    \[
        \begin{array}{l}
            (1-h\lambda)\varepsilon_{n+1}=\varepsilon_n\\
            \varepsilon_{n+1} = \dfrac{|\varepsilon_n|}{| (1-h\lambda)|}\leqslant |\varepsilon_n|
        \end{array}
    \]
    当$|1 - h\lambda |> 1$,即$h \geqslant 0$时,格式稳定
\end{itemize}
\subsection{线性多步法(1)}
基于数值积分的构造方法
\[
    \colorbox{yellow!50}{$y(x_{n+k})-y(x_{n-j})=\int_{x_{n-j}}^{x_{n+k}}f(x,y(x))\mathrm{d}x$}
\]
用$f(x,y(x))\approx P_{q}(x)=\sum_{i=0}^{q}f(x_{n-i},y_{n-i})L_{i}(x)$来估计,其中
\[
    L_{i}(x)=\prod_{\substack{l=0\\ l\neq i}}^{q}\frac{x-x_{n-l}}{x_{n-i}-x_{n-l}}.
\]
得到
\[
    \begin{aligned}
        y(x_{n+k})-y(x_{n-j}) &= \sum_{i=0}^{q}f(x_{n-i},y_{n-i})\int_{x_{n-j}}^{x_{n+k}}L_{i}(x)\mathrm{d}x\\
        &=\sum_{i=0}^{q}f(x_{n-i},y_{n-i})\int_{x_{n-j}}^{x_{n+k}}\prod_{\substack{l=0\\ l\neq i}}^{q}\frac{x-x_{n-l}}{x_{n-i}-x_{n-l}}\mathrm{d}x\\
        &\overset{x = x_n+th}{=}\sum_{i=0}^{q}f(x_{n-i},y_{n-i})\int_{x_{n-j}}^{x_{n+k}}\prod_{\substack{l=0\\ l\neq i}}^{q}\frac{(x_n+th)-(x_{n}-lh)}{(x_{n}-ih)-(x_{n}-lh)}\mathrm{d}x\\
        &=h\sum_{i=0}^{q}f(x_{n-i},y_{n-i})\colorbox{red!50}{$\int_{-j}^{k}\prod\limits_{\substack{l=0\\ l\neq i}}^{q}\dfrac{l+t}{l-i}\mathrm{d}t$}
    \end{aligned}
\]
线性多步法的计算公式
\begin{itemize}
    \item $k=1,j=0$时,得到\colorbox{red!50}{Adams显式方法}:
    \[
        y_{n+1}=y_{n}+h\sum_{i=0}^{q}\colorbox{red!50}{$\beta_{qi}$}f_{n-i}
    \]
    \[
        y_{n+1}=y_{n}+\frac{h}{24}(55f_{n}-59f_{n-1}+37f_{n-2}-9f_{n-3})
    \]
    \item $k=0,j=1$时,得到\colorbox{cyan!50}{Adams隐式方法}:
    \[
        y_n=y_{n-1}+h\sum_{i=0}^q\beta_{qi}f_{n-i}
    \]
    \[
        y_{n+1}=y_n+h[\beta_{q0}f_{n+1}+\beta_{q1}f_n+\cdots+\beta_{qq}f_{n-q+1}]
    \]
    \[
        y_{n+1}=y_{n}+\frac{h}{24}(9f_{n+1}+19f_{n}-5f_{n-1}+f_{n-2})
    \]
\end{itemize}
\begin{example}
    假设网格点等间距分布,空间步长为$h$,请采用数值积分方法基于$x_{n-1},x_n,x_{n+1}$构造一个隐式多步法。\Stars{5}{}
    \begin{solution}
        3个点,对2阶拉格朗日多项式换元积分,得到
        \[
            \begin{aligned}
                &\int_{x_{n}}^{x_{n+1}} f_{n-1}\dfrac{(x-x_{n})(x-x_{n+1})}{(x_{n-1}-x_{n})(x_{n-1}-x_{n+1})}+f_{n}\dfrac{(x-x_{n-1})(x-x_{n+1})}{(x_{n}-x_{n-1})(x_{n}-x_{n+1})}\\
                &+f_{n+1}\dfrac{(x-x_{n-1})(x-x_{n})}{(x_{n+1}-x_{n-1})(x_{n+1}-x_{n})} \mathrm{d}x\\
                &\overset{x = x_{n}+th}{=}h\int_{0}^{1}f_{n-1}\dfrac{t(t-1)}{-1\cdot (-2)}+f_{n}\dfrac{(t+1)(t-1)}{1\cdot (-1)}+f_{n+1}\dfrac{(t+1)t}{2\cdot 1}\\
                &=h\left[ \dfrac{-1}{12}f_{n-1} + \dfrac{2}{3}f_{n} + \dfrac{5}{12}f_{n+1} \right]
            \end{aligned}
        \]
        故而,隐式多步法可以构造为
        \[
            \begin{aligned}
                &y_{n+1}-y_n = h\left[ \dfrac{-1}{12}f_{n-1} + \dfrac{2}{3}f_{n} + \dfrac{5}{12}f_{n+1} \right]\\ 
                \Rightarrow & y_{n+1} = y_n + \dfrac{h}{12}\left[5f_{n+1} + 8f_{n}-f_{n-1}  \right]
            \end{aligned}
        \]
    \end{solution}
\end{example}
\begin{note}
    基于Taylor展开的构造方法

    \begin{itemize}
        \item 线性$k$步法的一般形式
        \[
            y_{n+1} = \sum_{k=0}^{r}\alpha_ky_{n-k}+h \sum_{k = -1}^{r}\beta_{k}f_{n-k}
        \]
    \end{itemize}
\end{note}

\begin{example}
    关于前面预测-校正算法,下列叙述正确的是?\Stars{4}{}
    \begin{enumerate}
        \item [\choice{1}{A}] 预测-校正算法是一个显式方法
        \item [\choice{1}{B}] 预测-校正算法是一个显式方法预测步和校正步的截断误差为同阶
        \item [\choice{}{C}] 预测-校正算法的整体截断误差是$O(h^5)$
        \item [\choice{1}{D}] 由于初始时刻无预测值和校正值,所以可令它们的差为零,然后多次校正
    \end{enumerate}
\end{example}
\subsection{方程组与刚性问题}
\begin{example}
    考虑方程组$\begin{cases}\frac{\mathrm{d}Y}{\mathrm{d}x}=JY,0\leq x\leq 1,\\Y(0)=(2,1,2)^\mathrm{T}\end{cases}$

    其中$J=\begin{pmatrix}-0.1&-49.9&0\\0&-50&0\\0&70&-30000\end{pmatrix}$
    \begin{solution}
        上述问题的解为
        \[
            \begin{cases}
                Y_1(x) = e^{-0.1x}+e^{-50x}\\
                Y_{2}(x) = e^{-50x}\\
                Y_3(x) = e^{-50x}+e^{-30000x}
            \end{cases}
        \]
        \colorbox{red!50}{不同的解的分量有不同的衰减速度}
    \end{solution}
\end{example}
考虑方程组
\[
    \dfrac{\mathrm{d}Y}{\mathrm{d}x} = JY + Z
\]
设$J$有互异的特征值$\lambda_k$,对应的特征向量为$C_{k}$,则方程组的通解为
\[
    Y = \sum_{k = 1}^mU_ke^{\lambda_k}C_{k} + V(x)
\]
其中$Q_k$为常数,$V(x)$为特解
\begin{definition}[刚性]
    上述线性方程组称为刚性的,若$\operatorname{Re}(\lambda_k)<0$,$\max |\operatorname{Re}(\lambda_k)|\gg \min |\operatorname{Re}(\lambda_k)$,此时$\frac{\max |\operatorname{Re}(\lambda_k)|}{ \min |\operatorname{Re}(\lambda_k)|}$称为刚性比
\end{definition}
\begin{example}
    一阶微分方程初值问题
    \[
        \begin{cases}
        u_1'=-0.01u_1-99.99u_2\\
        u_2'=-100u_2\\
        u_1(0)=2,u_2(0)=1.
        \end{cases}
    \]
    的刚性比\sol{10000}。\Stars{4}{}
    \begin{solution}
        \[
            \boldsymbol{A} = 
                \begin{bmatrix}
                    -0.01 & 99.99\\
                    0 & -100
                \end{bmatrix}
        \]
        特征值为$\lambda_1 = -0.01,\,\lambda_2 = -100$。故而刚性比为$\dfrac{100}{0.01} = 10000$
    \end{solution}
\end{example}
\subsection{边值问题的解法}
考虑两点边值问题
\[
    \begin{array}{c}
        y^{\prime\prime}=f(x,y,y^{\prime}),a<x<b,\\
        y(a)=\alpha\\
        y(b)=\beta 
    \end{array}
\]
\begin{note}
    试射法(shooting):
    将边值问题转化为初值问题求解,反复调整初始时刻的斜率$y'(a)$,使得初值问题的解能够命中$y(b) = \beta$.试射法也称打靶法
    设
    \[
        \begin{array}{cc}
            y_{1}''=f(x,y_{1},y_{1}') & y_{2}^{\prime\prime}=f(x,y_{2},y_{2}^{\prime}) \\ 
            y_{1}(a)=\alpha  & y_{2}(a)=\alpha \\
            y_{1}^{\prime}(a)=m_{1} & y_{2}'(a)=m_{2}
        \end{array}
    \]
    \[
        \begin{array}{c}
            c_1y_1(b) + c_2y_{1}(b) = y_{1}(b)\Rightarrow c_2 = \dfrac{\beta-y_1(b)}{y_{2}(b)-y_{1}(b)}\\
            c_1y_2(b) + c_2y_{2}(b) = y_{2}(b)\Rightarrow c_1 = \dfrac{\beta-y_2(b)}{y_{1}(b)-y_{2}(b)}\\
        \end{array}
    \]
    \[
        \begin{array}{c}
            y=c_{1}y_{1}+c_{2}y_{2}\\c_{1}+c_{2}=1\\c_{1}y_{1}(b)+c_{2}y_{2}(b)=\beta 
        \end{array}
    \]
\end{note}
\begin{note}
    有限差分法
    \[
        \begin{cases}y''=f(x,y,y'),a\leq x\leq b,\\y(a)=\alpha,y(b)=\beta.\end{cases}
    \]
    将求解区域$[a,b]$分成$N$等份,步长$h =\dfrac{b-a}{N}$网点$x_n = a+nh,n = 0,1,2,\cdots,N.$

    \[
        \begin{array}{ll}
            y''|_{x_n} = \colorbox{red!50}{$\dfrac{y_{n+1}-2y_{n}+y_{n-1}}{h^{2}}$} & y'|_{x_n} = \colorbox{cyan!50}{$\dfrac{y_{n+1}-y_{n-1}}{2h}$}+O(h^2)
        \end{array}    
    \]
    差分近似得到
    \[
        \begin{array}{c}
            \colorbox{red!50}{$\dfrac{y_{n+1}-2y_{n}+y_{n-1}}{h^{2}}$}=f(x_{n},y_{n},\colorbox{cyan!50}{$\dfrac{y_{n+1}-y_{n-1}}{2h}$})\\
            y_{0}=\alpha,y_{n}=\beta.
        \end{array}
    \]
    $N+1$个差分点,$N-1$个未知量,$N-1$个差分方程
\end{note}
 
