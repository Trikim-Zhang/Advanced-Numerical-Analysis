\section{矩阵的特征值和特征向量的计算}
\subsection{特征值理论}
\begin{theorem}[Gerschgorin圆盘定理]
关于$\boldsymbol{A}$有以下结论
\begin{enumerate}
    \item 设$\boldsymbol{A}=(a_{ij})_{n\times n}$,则$\boldsymbol{A}$的每一个特征值必属于下述某 个 圆 盘 ; $\mid \lambda - a_{ii}\mid \leq r_{i}= \sum _{j\neq i}\mid a_{ij}\mid$ $( j= 1, 2, \cdots , n)$
    \item 若$\boldsymbol{A}$的$m$圆盘组成并集$S($连通的)且与余下的$n-m$个圆盘是分离的(即不相交),则S内恰包含$m$个$\boldsymbol{A}$的特征值。特别,当S是由一个圆盘组成且与其他$n-1$个圆盘是分离的(即为孤立圆盘),则S中精确地包含$\boldsymbol{A}$的一个特征值。
\end{enumerate}
\end{theorem}
\begin{proof}
    设$\boldsymbol{x}\neq \boldsymbol{0}$为$\boldsymbol{A}$的特征向量,那么有
    \[
        \boldsymbol{Ax} = \lambda\boldsymbol{x}
    \]
    取$\boldsymbol{x}$中分量的绝对值最大的一个,设为$x_{i}$
    \[
        \begin{aligned}
            &(\lambda -a_{ii})x_{i} = \sum_{\substack{j = 1\\j\neq i}}a_{ij}x_{j}\\
            \Rightarrow & |\lambda -a_{ii}||x_{i}| = |\sum_{\substack{j = 1\\j\neq i}}a_{ij}||x_{j}|  \\
            \Rightarrow & |\lambda -a_{ii}| = |\sum_{\substack{j = 1\\j\neq i}}a_{ij}|\left|\frac{x_{j}}{x_{i}}\right|< r_{i}  \\
        \end{aligned}
    \]
    证毕!
\end{proof}
\begin{theorem}
    设$\boldsymbol{A}\in\mathbb{R}^{n\times n}$为对称矩阵,其特征值为$\lambda_1\geq\lambda_2\geq\cdots\geq\lambda_n$,其对应的特征向量 $\boldsymbol{x}_1, \boldsymbol{x}_2, \cdots , \boldsymbol{x}_n$组成规范化正交组,则
\begin{enumerate}
    \item  $\lambda _n\leq \dfrac {( \boldsymbol{A}\boldsymbol{x}, \boldsymbol{x}) }{( \boldsymbol{x}, \boldsymbol{x}) }\leq \lambda _1$ $( \forall \boldsymbol{x}\in \mathbb{R} ^n, \boldsymbol{x}\neq 0)$
    \item $\lambda_1=\max\limits_{\boldsymbol{x}\in\mathbb{R}^n}R(\boldsymbol{x})$
    \item $\lambda_n=\min\limits_{\boldsymbol{x}\in\mathbb{R}^n}R(\boldsymbol{x})$
\end{enumerate}
\end{theorem}
\subsection{幂法}
\subsubsection{幂法}
一种计算矩阵主特征值(按模最大特征值) 及其特征向量的迭代法。设$\boldsymbol{A}=(a_{ij})\in\mathbb{R}^{n\times n}$,有一组线性无关的特征向量组
\[
    \boldsymbol{A}\boldsymbol{x}_i=\lambda_i\boldsymbol{x}_i\quad(i=1,2,\cdots,n)    
\]
其中$\{\boldsymbol{x}_1,\boldsymbol{x}_2,\cdots,\boldsymbol{x}_n\}$线性无关,且满足:$\mid\lambda_1\mid>\mid\lambda_2\mid\geq\cdots\geq\mid\lambda_n\mid$

\begin{note}
    基本思想:任取初始向量$\boldsymbol{v}_0\in\mathbb{R}^n$且$\boldsymbol{v}_0\neq0$
    \[
        \left\{
            \begin{array}{l}
                \boldsymbol{v}_1=\boldsymbol{A}\boldsymbol{v}_0\\
                \boldsymbol{v}_2=\boldsymbol{A}\boldsymbol{v}_1=\boldsymbol{A}^2\boldsymbol{v}_0\\
                \vdots\\
                \boldsymbol{v}_{k+1}=\boldsymbol{A}\boldsymbol{v}_k=\boldsymbol{A}^{k+1}\boldsymbol{v}_0\\
                \vdots
            \end{array}
            \right.
    \]
    设$\boldsymbol{v}_0 = \sum\limits_{i = 1}^{n}\alpha_{i} \boldsymbol{x}_{i}$,$\boldsymbol{v}_{k} = \boldsymbol{A}^{k}\boldsymbol{v}_0 = \boldsymbol{A}^{k}\left( \sum\limits_{i = 1}^{n}\alpha_{i} \boldsymbol{x}_{i} \right) = \sum\limits_{i = 1}^{n}\alpha_{i}\lambda_{i}^{k}\boldsymbol{x}_{i}$,两边同时除以$\lambda_{1}^{k}$
    \[
        \dfrac{\boldsymbol{v}_{k}}{\lambda_{1}^{k}} = \alpha_{1}\boldsymbol{x}_{1} + \sum\limits_{i = 2}^{n}\alpha_{i}\left( \dfrac{\lambda_{i}}{\lambda_{1}} \right)^{k}\boldsymbol{x}_{i}\approx \alpha_1\boldsymbol{x}_1
    \]
    有
    \[
        \begin{array}{l}
            \boldsymbol{v}_{k+1}\approx \alpha_1\lambda_1^{k+1}\boldsymbol{x}_1\\
            \boldsymbol{v}_{k}\approx \alpha_1\lambda_1^k\boldsymbol{x}_1
        \end{array}
    \]
    \[
        \lim_{k\to\infty}\frac{\boldsymbol{v}_k}{\lambda_1^k}=\alpha_1\boldsymbol{x}_1\quad\lim_{k\to\infty}\frac{(\boldsymbol{v}_{k+1})_i}{(\boldsymbol{v}_k)_i}=\lambda_1
    \]
\end{note}
\subsubsection{改进幂法}
设$\boldsymbol{u}_0=\boldsymbol{v}_0\neq \boldsymbol{0}(\alpha_1\neq0)$

迭代$\boldsymbol{v}_k= \boldsymbol{Au}_{k-1}$,$\mu _{k}= \max ( \boldsymbol{v}_{k})$,$k= 1, 2, \cdots$

规范化:$\boldsymbol{u}_k=\boldsymbol{v}_k/\mu_k$

% Table generated by Excel2LaTeX from sheet '改进幂法'
\begin{table}[htbp]
    \centering
    \begin{tabular}{cc}
        迭代序列 & 规范化序列 \\
        $\boldsymbol{v}_{1} = \boldsymbol{Au}_{0} =  \boldsymbol{Av}_{0}$ & $\boldsymbol{u}_{1} = \dfrac{\boldsymbol{Av}_{0}}{\max(\boldsymbol{Av}_{0})}$ \\
        $\boldsymbol{v}_{2} = \dfrac{\boldsymbol{A}^{2}\boldsymbol{v}_{0}}{\max(\boldsymbol{Av}_{0})}$ & $\boldsymbol{u}_{2} = \dfrac{\boldsymbol{A}^{2}\boldsymbol{v}_{0}}{\max(\boldsymbol{A}^{2}\boldsymbol{v}_{0})}$ \\
        $\cdots$ & $\cdots$ \\
        $\boldsymbol{v}_{k} = \dfrac{\boldsymbol{A}^{k}\boldsymbol{v}_{0}}{\max(\boldsymbol{A}^{k-1}\boldsymbol{v}_{0})}$ & $\boldsymbol{u}_{k} = \dfrac{\boldsymbol{A}^{k}\boldsymbol{v}_{0}}{\max(\boldsymbol{A}^{k}\boldsymbol{v}_{0})}$ \\
    \end{tabular}%
\end{table}%  
仍然设$\boldsymbol{v}_0 = \sum\limits_{i = 1}^{n}\alpha_{i} \boldsymbol{x}_{i}$,那么
\[
    \boldsymbol{u}_{k} = \dfrac{\sum\limits_{i = 1}^{n}\alpha_i \lambda_{i}^{k}\boldsymbol{x}_{i}}{\max\left( \sum\limits_{i = 1}^{n}\alpha_i \lambda_{i}^{k}\boldsymbol{x}_{i} \right)} \overset{\text{同时除以}\lambda_{1}^{k}}{=} \dfrac{\alpha_1 \boldsymbol{x}_{1}+\sum\limits_{i = 2}^{n}\alpha_i (\frac{\lambda_{i}}{\lambda_{1}})^{k}\boldsymbol{x}_{i}}{\max\left(\alpha_1 \boldsymbol{x}_{1}+ \sum\limits_{i = 2}^{n}\alpha_i (\frac{\lambda_{i}}{\lambda_{1}})^{k} \boldsymbol{x}_{i} \right)} = \dfrac{\boldsymbol{x}_{1}}{\max\left( \boldsymbol{x}_{1} \right)},\,k\to\infty 
\]
同时
\[
    \mu_{k} = \max\left( \boldsymbol{v}_{k} \right) = \lambda_1
\]
\begin{theorem}[改进幂法]
    设(1)$A=(a_{ij})\in\mathbb{R}^{n\times n}$有$n$个线性无关的特征向量;(2)设$A$的特征值满足$:|\lambda_1|>|\lambda_2|\geq\cdots\geq|\lambda_n|$且$ Ax_{i}= \lambda _{i}x_{i}$ $( i= 1, 2, \cdots , n)$;(3)$\{u_{k}\},\{\nu_{k}\}$由改进幂法得到,则有:
    \begin{enumerate}
        \item $\lim\limits_{k\to \infty}\boldsymbol{u}_{k} = \dfrac{\boldsymbol{x}_{1}}{\max\left( \boldsymbol{x}_{1} \right)}$
        \item $\mu_{k} = \max\left( \boldsymbol{v}_{k} \right) = \lambda_1$
        \item 且收敛速度$r = \left|\dfrac{\lambda_2}{\lambda_1}\right|$确定。
    \end{enumerate}
\end{theorem}
% \subsection{反幂法}